\documentclass{article}
\usepackage{listings}
\usepackage{color}

\definecolor{dkgreen}{rgb}{0,0.6,0}
\definecolor{gray}{rgb}{0.5,0.5,0.5}
\definecolor{mauve}{rgb}{0.58,0,0.82}

\lstset{frame=tb,
  language=Python,
  aboveskip=3mm,
  belowskip=3mm,
  showstringspaces=false,
  columns=flexible,
  basicstyle={\small\ttfamily},
  numbers=none,
  numberstyle=\tiny\color{gray},
  keywordstyle=\color{blue},
  commentstyle=\color{dkgreen},
  stringstyle=\color{mauve},
  breaklines=true,
  breakatwhitespace=true,
  tabsize=3
}

\title{NumPy - Introduction}
\author{Michał Tracewicz}
\date{2020-03-07}
\begin{document}
\pagenumbering{gobble}
\maketitle
\newpage
\pagenumbering{arabic}
\section{What is NumPy?}
NumPy is general purpose Python package used for array-processing.
It provides a high-performance multidimensional array object, and tools for working with these arrays.
\section{Basics}
\subsection{Introduction}
In NumPy the main way of representing data are multi dimensional arrays.
The are represented by ndarray class. It represents table of elements (all of which are of the same type).
Those arrays are indexed by tuples of positive integeres
Number of dimensions for given array is called rank.
Tuple of integers representing size of array along each dimension is called shape.
\subsection{Creating arrays}
Example code presenting how to create ndarrays:
\begin{lstlisting}
import numpy as np

# Creating a rank 1 Array saved in column first way
arr = np.array([1, 2, 3],order = 'F')
print("Array with Rank 1: \n",arr)

# Creating a rank 2 Array
arr = np.array([[1, 2, 3],
                [4, 5, 6]])
print("Array with Rank 2: \n", arr)

# Get array rank
print(arr.ndim)

# Creating an array from tuple
arr = np.array((1, 3, 2))
print("\nArray created using "
        "passed tuple:\n", arr)

# Creating an array filled with 0
arr = np.zeros(10)
print("\nArray created using "
        "zeros:\n", arr)

# Creating an array filled with 1
arr = np.ones(10)
print("\nArray created using "
        "ones:\n", arr)

# Creating an array filled with from 2.0 to 10.0 containing 5 elements
arr = np.linspace(2, 10, 5)
print("\nArray created using "
        "linspace:\n", arr)

# Creating an array filled random numbers (max numebr is 9, because it will generate numebrs lower then given n))
np.random.seed(0)
arr = np.random.randint(10, size = 6)
print("\nArray created using "
        "random:\n", arr)
\end{lstlisting}
\subsection{Accessing elements of arrays}
Elements of those arrays are accesed using square brackes.
When using ndarrays one can also use slices.
\begin{lstlisting}
import numpy as np

# Initial Array
arr = np.array([[-1, 2, 0, 4],
                [4, -0.5, 6, 0],
                [2.6, 0, 7, 8],
                [3, -7, 4, 2.0]])
print("Initial Array: ")
print(arr)
  
# Printing a range of Array
# with the use of slicing method
sliced_arr = arr[:2, ::2]
print ("Array with first 2 rows and"
    " alternate columns(0 and 2):\n", sliced_arr)

# Printing elements at
# specific Indices
Index_arr = arr[[1, 1, 0, 3],
                [3, 2, 1, 0]]
print ("\nElements at indices (1, 3), "
    "(1, 2), (0, 1), (3, 0):\n", Index_arr)

x = np.array([[1, 2], [3, 4], [5, 6]])
y = x[[0,1,2], [0,1,0]] 
# First array contains indexes in first dimension, second array contains indexes in second dimsenision and so on
# This means that example above gives use elements at indexes (0,0),(1,1),(2,0)
print(y)

# NumPy also allows use to get only the elements which meet some criteria
# Elements bigger then 5
print(a[a>5])
# This omits NaN (Not a number) elements
print a[~np.isnan(a)]
\end{lstlisting}
\subsection{Iteration over array}
\begin{lstlisting}
import numpy as np
a = np.arange(0,60,5)
a = a.reshape(3,4)

print(f'Original array is:\n{a}')
print(f'Transpose of the original array is:\n{a.T}')

# Both this and next for will print the same output as iteration
# is done according to how array is storred in memory
for x in np.nditer(a):
        print(f"{x} ")

for x in np.nditer(a.T):
        print(f"{x} ")

# This however will always iterate in colum first way
for x in np.nditer(a, order = 'F'):
        print(f"{x} ")

# User can modify elements when iterating by setting a flag
for x in np.nditer(a, op_flags=['readwrite']):
    x[...] = 2*x+1
print(f'Modified array is:\n{a}')
\end{lstlisting}
\subsection{Data types}
Every ndarray has associated data type, known as dtype.
It provides information about arrays layout.
This example should ilustrate how NumPy tries to guess data 
and how user can force it to use particular one.
\begin{lstlisting}
import numpy as np

# Integer datatype
# guessed by Numpy
x = np.array([1, 2])
print("Integer Datatype: ",x.dtype)

# Float datatype
# guessed by Numpy
x = np.array([1.0, 2.0])
print("\nFloat Datatype: ",x.dtype)

# Forced Datatype
x = np.array([1, 2], dtype = np.int64)
print("\nForcing a Datatype: ",x.dtype)
\end{lstlisting}
\section{Operations}
NumPy provides a set of usefull buildin functions.

\begin{lstlisting}
import numpy as np

# Defining Array 1
a = np.array([[1, 2],
              [3, 4]])
  
# Defining Array 2
b = np.array([[4, 3],
              [2, 1]])
                
# Adding 1 to every element
print ("Adding 1 to every element:", a + 1)
  
# Subtracting 2 from each element
print ("\nSubtracting 2 from each element:", b - 2)
  
# sum of array elements
# Performing Unary operations
print ("\nSum of all array "
        "elements: ", a.sum())

# Adding two arrays
# Performing Binary operations
print ("\nArray sum:\n", a + b)

# Muliplying two arrays
print ("\nArray sum:\n", a * b)

# Gettig dot product
print ("\nArray sum:\n", a @ b)

# Transposition
print ("\nMatrix transposition:\n", a,a.T)

# Broadcasting
# In some situations NumPy can help use perform operations which would normaly be imposible.
# When arrays are not of the same rank, NumPy can sometimes expand smaller one to allow the operation.
# This expansion is called broadcasting.
# In this example array b will be broadcasted to match array a
a = np.array([[0.0,0.0,0.0],[10.0,10.0,10.0],[20.0,20.0,20.0],[30.0,30.0,30.0]]) 
b = np.array([1.0,2.0,3.0])

print(f'First array:\n{a}')

print(f'Second array:\n{b}')

print(f'First + Second array:\n{a+b}')

# Sorting
# NumPy allows to chose from quicksort, mergesort, heapsort algorithms (passed as kind argument)
np.sort(a,axis = 0,kind = "quicksort")
\end{lstlisting}
\section{List of NumPy methods:}
\begin{itemize}
  \item all()
  \item any()
  \item take()
  \item put()
  \item apply\_along\_axis()
  \item apply\_over\_axes()
  \item argmin()
  \item argmax()
  \item nanargmin()
  \item nanargmax()
  \item amax()
  \item amin()
  \item insert()
  \item delete()
  \item append()
  \item around()
  \item flip()
  \item fliplr()
  \item flipud()
  \item triu()
  \item tril()
  \item tri()
  \item empty()
  \item empty\_like()
  \item zeros()
  \item zeros\_like()
  \item ones()
  \item ones\_like()
  \item full\_like()
  \item diag()
  \item diagflat()
  \item diag\_indices()
  \item asmatrix()
  \item bmat()
  \item eye()
  \item roll()
  \item identity()
  \item arange()
  \item place()
  \item extract()
  \item compress()
  \item rot90()
  \item tile()
  \item reshape()
  \item ravel()
  \item isinf()
  \item isrealobj()
  \item isscalar()
  \item isneginf()
  \item isposinf()
  \item iscomplex()
  \item isnan()
  \item iscomplexobj()
  \item isreal()
  \item isfinite()
  \item isfortran()
  \item exp()
  \item exp2()
  \item fix()
  \item hypot()
  \item absolute()
  \item ceil()
  \item floor()
  \item degrees()
  \item radians()
  \item npv()
  \item fv()
  \item pv()
  \item power()
  \item float\_power()
  \item log()
  \item log1()
  \item log2()
  \item log10()
  \item dot()
  \item vdot()
  \item trunc()
  \item divide()
  \item floor\_divide()
  \item true\_divide()
  \item random.rand()
  \item random.randn()
  \item ndarray.flat()
  \item expm1()
  \item bincount()
  \item rint()
  \item equal()
  \item not\_equal()
  \item less()
  \item less\_equal()
  \item greater()
  \item greater\_equal()
  \item prod()
  \item square()
  \item cbrt()
  \item logical\_or()
  \item logical\_and()
  \item logical\_not()
  \item logical\_xor()
  \item array\_equal()
  \item array\_equiv()
  \item sin()
  \item cos()
  \item tan()
  \item sinh()
  \item cosh()
  \item tanh()
  \item arcsin()
  \item arccos()
  \item arctan()
  \item arctan2()
  \item flatten
  \item transpose
  \item rollaxis
  \item swapaxes
  \item expand\_dims
  \item squeeze
  \item concatenate
  \item stack
  \item htsack
  \item vstack
  \item split
  \item hsplit
  \item vsplit
\end{itemize}
\end{document}